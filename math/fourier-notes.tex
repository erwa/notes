\documentclass[11pt]{article}
\usepackage[export]{adjustbox}
\usepackage{amsmath}
\usepackage{amssymb}
\usepackage{enumitem}
\usepackage{hyperref}
\usepackage[margin=1.0in]{geometry}
\usepackage{graphicx}
\usepackage{listings}
\usepackage{makecell}
\usepackage{xcolor}
\newcommand{\mat}[1]{\begin{bmatrix}#1\end{bmatrix}}

\newcommand{\eq}[1]{\begin{align*}#1\end{align*}}
\newcommand{\link}[2]{{\color{blue}\href{#1}{#2}}}
\newcommand{\norm}[1]{\left\lVert#1\right\rVert}

\title{Fourier transform notes}

\setlength{\parskip}{1em}
\begin{document}
\maketitle

\section{Discrete Fourier Transform}

Given a sequence $\{x_n\} := x_0, x_1, ..., x_{N-1}$, by definition it converts it into another sequence $\{X_k\}:= X_0,X_1,...,X_{N-1}$ by applying the following transformation (the discrete Fourier transform:
\eq{
X_k = X(\omega^k) &= \sum_{n=0}^{N-1} x_n \cdot \omega^{kn} \\
&= \sum_{n=0}^{N-1} x_n \cdot e^{-\frac{i2\pi}{N}kn}
}
Note that calculating each term of $X_k$ requires combining all terms in $x_n$. The calculation can be done in $O(n \log n)$ time using the Fast Fourier Transform.

The discrete Fourier basis consists of the vectors
\eq{
u_k = \left[\begin{array}{c|c}
e^{\frac{i2\pi}{N}kn} & n = 0,1,...,N-1 \\
\end{array}\right]
}
The vectors $u_k$ are orthogonal to one another (though they are not unit vectors).

Inverse discrete Fourier transform is defined by
\eq{
x_n = x(\omega^n) &= \frac{1}{N}\sum_{k=0}^{N-1} X_k \cdot \omega^{kn} \\
&= \frac{1}{N}\sum_{k=0}^{N-1} X_k \cdot e^{i2\pi kn/N}
}
Note that calculating a term $x_n$ involves combining all $N$ $X_k$ terms. This transform can also be done in $O(n \log n)$ time using the Fast Fourier Transform.

Convolution of two sequences can be done as multiplication in the frequency domain.

See \url{https://en.wikipedia.org/wiki/Discrete_Fourier_transform} for more details.
\end{document}