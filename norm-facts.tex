\documentclass[11pt]{article}
\usepackage[export]{adjustbox}
\usepackage{amsmath}
\usepackage{amssymb}
\usepackage{enumitem}
\PassOptionsToPackage{hyphens}{url}\usepackage{hyperref}
\usepackage[margin=1.0in]{geometry}
\usepackage{graphicx}
\usepackage{listings}
\usepackage{makecell}
\usepackage{xcolor}
%\usepackage[hyphenbreaks]{breakurl}

\newcommand{\eq}[1]{\begin{align*}#1\end{align*}}
\newcommand{\norm}[1]{\left\lVert#1\right\rVert}

\title{Facts about matrix and vector norms}

\setlength{\parskip}{1em}
\begin{document}
\maketitle

\section{Vector norm properties}

\begin{itemize}
\item $\norm{x + y} \leq \norm{x} + \norm{y}$ (subadditive property or triangle inequality)
\end{itemize}

\section{General matrix norm properties}

Must hold for all matrix norms.

\begin{itemize}
\item $\norm{A + B} \leq \norm{A} + \norm{B}$ (sub-additive or satisfying triangle inequality)
\end{itemize}

\section{Frobenius Norm}

Frobenius norm of a matrix is square root of the sum of squares of all the elements of the matrix. It's like Euclidean norm extended to a matrix.

\subsection{Properties}

\begin{itemize}
\item
\eq{
\norm{AB}_F^2 \leq \norm{A}_F^2 \norm{B}_F^2
}
Proof: \url{https://math.stackexchange.com/questions/1393301/frobenius-norm-of-product-of-matrix}

\item Count-Sketch preserves Frobenius norm with high probability:
\eq{
\Pr[\norm{SA}_F^2 = (1 \pm \epsilon) \norm{A}_F^2] > 3/4
}

\item $\norm{A}_2 \leq \norm{A}_F$. The operator norm is less than or equal to the Frobenius norm.

Proof: \url{https://math.stackexchange.com/questions/252819/why-is-frobenius-norm-of-a-matrix-greater-than-or-equal-to-the-2-norm}
\end{itemize}

\section{Operator Norm}
The operator norm of $A$, $\norm{A}_2$, is defined as follows:
\eq{
\norm{A}_2 = \sup_{\norm{x}_2=1} \norm{Cx}_2
}

If we use the Euclidean norm on the right, then the operator norm of $A$ is just the largest singular value of $A$, which is the square root of the largest eigenvalue of $A^T A$ ($A^* A$ if you want to be strict).

The value of the operator norm depends on what vector norm you use. However, we do have
\eq{
\text{operator norm of $A$} \leq \sigma_{max} = \text{spectral norm of $A$}
}

Ref: \url{https://math.stackexchange.com/questions/482170/2-norm-vs-operator-norm}

If you use the Euclidean vector norm, then the operator norm is the same as the spectral norm. Ref: \url{https://en.wikipedia.org/wiki/Matrix_norm#Special_cases}

Another definition of operator norm given vectors $x, y$:
\eq{
\norm{A}_2 = \sup_{\norm{x}=\norm{y}=1} |x^T A y|
}

\section{Spectral Norm}

The spectral norm of $A$ is its largest singular value. It's a measure of how much $A$ can stretch a vector.

\end{document}